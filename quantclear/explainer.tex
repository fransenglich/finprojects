% TODO:
% * Plot the tangency portfolio
% * Plot the Capital Allocation Line (CAL)
% * Document that mean/variance is historical and how much.
% * Make it solid. Understand and check it.
% * Incorporate Oscar's reply.
% * Discuss that our time series aren't long.

\documentclass[a4paper]{article}

\usepackage[utf8]{inputenc}

\usepackage{csquotes}

\usepackage[
    backend=biber,
    style=authoryear-icomp,
    natbib=true,
    url=false,
    doi=true,
    eprint=false
]{biblatex}

\addbibresource{references.bib}

\usepackage{graphicx}
\usepackage{float}
\usepackage{amsmath}
\usepackage[english]{babel}
\usepackage[bookmarks=true]{hyperref}
\usepackage{color}
\usepackage{listings}

\usepackage[
    type={CC},
    modifier={by-nc-sa},
    version={3.0},
]{doclicense}

\usepackage{xcolor}
% New colors defined below.
\definecolor{codegreen}{rgb}{0,0.6,0}
\definecolor{codegray}{rgb}{0.5,0.5,0.5}
\definecolor{codepurple}{rgb}{0.58,0,0.82}
\definecolor{backcolour}{rgb}{0.95,0.95,0.92}

%Code listing style named "mystyle"
\lstdefinestyle{mystyle}{
%  backgroundcolor=\color{backcolour},
  commentstyle=\color{codegreen},
  keywordstyle=\color{magenta},
  numberstyle=\tiny\color{codegray},
  stringstyle=\color{codepurple},
  basicstyle=\ttfamily\footnotesize,
  breakatwhitespace=false,
  breaklines=true,
  keepspaces=true,
  numbers=left,
  numbersep=5pt,
  showspaces=false,
  showstringspaces=false,
  showtabs=false,
  tabsize=2
}

%"mystyle" code listing set
\lstset{style=mystyle}

\def\documenttitle{Quant clearly clarified}

\title{\documenttitle}
\date{\today}
\author{Frans Englich\texorpdfstring \\
        \href{mailto:fenglich@fastmail.fm}{fenglich@fastmail.fm}}

\hypersetup{
    pdfsubject = {\documenttitle},
    pdftitle = {\documenttitle}
}

\begin{document}

\maketitle

This paper is a general, schematic overview of the quant trading process. Source is a mail from Johannes.

First we have our time series of prices of an asset, $p_t$, that we transform into a time series of returns (percentages):

\begin{equation*}
    r_t = \frac{p_{t + dt}}{p_{t - 1}}
\end{equation*}

Next, we volatility normalize it, "flatten" it:

\begin{equation*}
    y_t = \frac{r_t}{\sigma_t}
\end{equation*}

The volatility can be computed anyhow as per taste; EWMA, or more fancy like GARCH, etc.

Now we model it linearly:

\begin{equation*}
    y_t = \beta*x_t + \epsilon
\end{equation*}

$y_t$ is a time series of returns (percentages) that we normalised before. $x_t$, however can be any kind of data. The choice of $x_t$ is our clever competitive advantage that makes our trading.

We run the regression in previous equation and get $\beta$, a constant. This can be a bit confusing. When running rolling we get several betas, one for each regression, but it's not "a time series of betas."

Now we have 2 known components, $x_t$ and $\beta$, which we use and run in a regression to get $\hat{y}$:

\begin{equation*}
    \hat{y}_t = \beta*x_t   
\end{equation*}

Now our process has produced another time series, $\hat{y}$, which is our prediction of our normalised returns, $y_t$.

The question is, is it any good? We can first look at (MATLAB code):

\begin{lstlisting}[language=Matlab, mathescape=true]
cumsum($\hat{y}$ * $y_t$)
\end{lstlisting}

It can be plotted with $y_t$ for comparison, they are similar in nature.

The prediction can then further be analysed to show whether it's profitable, by adjusting for the trading cost.

Another perspective is more traditional science, and look at the regression we ran to produce $\hat{y}$. What is its statistical power? What's its $R^2$ value? And so forth.

\end{document}
